\documentclass[a4paper,11pt]{article} 
% Load some standard packages
\usepackage{amsmath,parskip,fullpage,natbib,bm} 

% Load tikz for the tree diagram
\usepackage{tikz}
\usetikzlibrary{trees,shapes}

% Define my own commands
\newcommand{\info}{{\cal I}}
\newcommand{\E}{\text{E}}

% Bibliography style
\bibliographystyle{agsm}

\begin{document}  
\title{Estimation of LMMs}
\author{Lucas}
\maketitle

\section{Introduction}

In this part we want to estimate values for the fixed effects coefficient $\beta$ and use the data to model $\sigma^2$ and the matrix $D$.We can generalize to a mixed effect model with a vector $\gamma$ of q random effects with associated model matrix Z which has dimension $n *q$.Then we can model the response y,given the value of the random effects as:
\begin{equation}
\bm{y}=\bm{X}\beta + \bm{Z}\gamma +\epsilon	
\end{equation}
If we further assume that the random effects $\gamma \sim N(0,\epsilon^2D)$, then $var(y) = var(Z\gamma)+var(\epsilon) = \sigma^2ZDZ^T + \sigma^2I$ and we can get the unconditional distribution of y as:
\begin{equation}
\bm{y} \sim N(\bm{X}\beta,\sigma^2(I+ZDZ^T))
\end{equation}
For simplicity,$\sigma^2(I+ZDZ^T)$ will be replaced by $H(\theta)$ ,where $H(\theta)$ is a general covariance matrix parametrized by $\theta$.


In matrix notation,denote $\bm{y}$=$\left[\begin{array}{c}
y_1 \\
y_2 \\
$\vdots$ \\
y_M \end{array} \right]$,      $\bm{X}$ = $ \left[ \begin{array}{c}
X_1 \\
X_2 \\
$\vdots$ \\
X_M \end{array} \right]$,      $\bm{Z}$ = $ \left[ \begin{array}{cccc}
Z_1 & 0 & \cdots & 0 \\
0 & Z_2 & \cdots& 0 \\
\vdots & \vdots & \ddots& \vdots\\
0 & 0& \cdots & Z_M \end{array} \right]$,	$\bm{\gamma} $= $ \left[ \begin{array}{c}
\gamma_1 \\
\gamma_2 \\
$\vdots$ \\
\gamma_M \end{array} \right]$,      $\epsilon$ = $ \left[ \begin{array}{c}
\epsilon_1 \\
\epsilon_2 \\
$\vdots$ \\
\epsilon_M \end{array} \right]$. 


Thus, with $\bm{H}(\theta$) being a diagonal matrix we get \begin{equation}
	\bm{y}\sim\mathcal{N}\biggl( \bm{X}\beta,\bm{H}(\theta)= \left[ \begin{array}{cccc}
	H_1(\theta) & 0 & \cdots & 0 \\
	0 & H_2(\theta)& \cdots& 0 \\
	\vdots & \vdots & \ddots& \vdots\\
	0 & 0& \cdots & H_M(\theta) \end{array} \right] \biggr),	
\end{equation}
\section{Known Variance-covariance Parameters}
Assuming that $H(\theta)$ are known, we can just maximize the joint log-likelihood of $y,\gamma$ with respect to $\beta$ and $\gamma$ and get the estimators.
\begin{equation}
\widehat{\beta}=\bm{X}^T\bm{H}^{-1}\bm{X})^{-1}\bm{X}^T\bm{H}^{-1}\bm{y}
\end{equation}
\begin{equation}
\widehat{\gamma}= DZ^TH^{-1}(y-X\widehat{\beta})
\end{equation}y
This implies that$\widehat{\beta}$ is a weighted least squares estimator with the inverse of the $H(\theta) = \sigma^2(I+ZDZ^T)$ of $\beta$ as the weight matrix.
\section{Unknown Variance-covariance Parameters}
This section describes the maximum likelihood (ML) and restricted maximum likelihood(REML) approach for the estimating unknown parameters $\theta$.
\subsection{Maximum Likelihood Estimation of $\theta$}
ML estimation of $\theta$ is based on the likelihood of the marginal model $y ~ N(X\beta,H(\theta))$
The corresponding log-likelihood,up to additive constants, is
\begin{equation}
log L(\beta,\theta|y,X) =l(\beta,\theta|y,X) = - \frac{1}{2}[log|H(\theta)| + (y - X\beta)^TH(\theta)^{-1}(y - X\beta)].
\end{equation}
The maximization of $l(\beta,\theta)$ with respect to $\beta$(while holding $\theta$ fixed)gives
\begin{equation}
\widehat \beta (\theta) = (X^TH(\theta)^{-1}X)^{-1}X^TH(\theta)^{-1}y
\end{equation}
Inserting $\widehat \beta (\theta)$ in $l(\beta,\theta)$ reults in the profile log-likelihood
\begin{equation}
l_p(\theta) = -\frac{1}{2}[log|H(\theta)| + (y-X\widehat\beta(\theta)^TH(\theta)^{-1}(y - X\widehat\beta(\theta)))]
\end{equation}
The maximization of $l_p(\theta)$ with respect to $\theta$ provides the ML estimator $\widehat\theta_{ML}$.
Refer to Harville(1977) for explicit formulae and algorithms.
\subsection{Restricted Maximum Likelihood Estimation of $\theta$}
Restricted maximum likelihood estimation of $\theta$ is often based on the resticted log-likelihood
\begin{equation}
l_R(\theta) = log(|L(\beta,\theta)d\beta)
\end{equation}
integrating out $\beta$ from the likelihood.It can be shown that restricted log-likelihood is 
\begin{equation}\label{eq:2} 
\begin{split}
{\mathcal{L}}_R (\theta| A^T y) &= -\frac{1}{2}\text{log det}H-\frac{1}{2}\text{log det} X^TH^{-1}X-\frac{1}{2}(y-X\widehat{\beta})^TH^{-1}(y-X\widehat{\beta}) \\
&= -\frac{1}{2}\text{log det}\bm{H}-\frac{1}{2}\text{log det}\bm{X}^T\bm{H}^{-1}\bm{X}-\frac{1}{2}(\bm{y}-\bm{X}\widehat{\beta})^TH^{-1}(\bm{y}-\bm{X}\widehat{\beta}) \\
&= -\frac{1}{2}\text{log}\prod_{i=1}^{M} \text{det}H_i-\frac{1}{2}\text{log}\prod_{i=1}^{M} \text{det} X_i^TH_i^{-1}X_i-\frac{1}{2}\sum_{i=1}^{M}(y_i-X_i\widehat{\beta})^TH_i^{-1}(y_i-X_i\widehat{\beta})\\
&= -\frac{1}{2}\sum_{i=1}^{M} \text{log det}H_i-\frac{1}{2}\sum_{i=1}^{M} \text{log det} X_i^TH_i^{-1}X_i-\frac{1}{2}\sum_{i=1}^{M}(y_i-X_i\widehat{\beta})^TH_i^{-1}(y_i-X_i\widehat{\beta})
\end{split}
\end{equation}
This equation can be used to get estimates of $\theta$ by maximizing it w.r.t. $\beta$, $\sigma$ and D.\footnote{for computational Details check appendix}
As shown previously??! 
\begin{equation}
\begin{split}
\widehat{\beta}&=(\bm{X}^T\bm{H}^{-1}\bm{X}^T\bm{H}^{-1}\bm{y})\\
&=\bigl(\sum_{i=1}^{M}X_i^TH_i^{-1}X_i\bigr)^{-1}\sum_{i=1}^{M}X_i^TH_i^{-1}y_i.
\end{split}
\end{equation}
Reduction of the bias of $\widehat\theta_{ML}$ is the main reason for preferring $\widehat\theta_{REML}$ in LMMs as an estimator for $\theta$

\subsection{why ML estimator is biased, REML not}
In the classical linear model,the estimator 
\begin{equation}
    {\hat{\sigma}}^2_{unbiased} &= \frac1{N-K}(y-X\hat{\beta})^T(y-X\hat{\beta})
\end{equation}
While using maximum likelihood, we can get$ E[{\hat{\sigma}}^ 2] = \frac{N-K}{N}\sigma^2<\sigma^2$.This implies that the ML estimator for $\sigma^2$ is biased.The reason of bias in LMMs maximum likelihood is that we neglect the loss of \textbf{degree of freedom}(DoF) for estimating $\beta$.

Generally, estimation bias in variance components originates from the DoF loss in estimating mean components. If we estimated variance components with true mean component values, the estimation would be unbiased. 
The intuition behind ReML is to maximize a modified likelihood that is free of mean components instead of the original likelihood as in ML.
If vector $a$ is orthogonal to all columns of X, i.e., $a^T X = 0 $, then $a^T y$ is known as an error contrast.We can find at most $N-k$ such vectors that are linearly independent\footnote{ Imagine a plane spanned by $k = 2$ linearly independent vectors in three-dimensional space $(N = 3)$. We can find at most $N - k = 1$ vector (passing the origin) orthogonal to the plane.}. Define $A =\begin{bmatrix} a_1 a_2 ... a_(N-k) \end{bmatrix}$. It follows that $A^T X = 0$ and $E{A^T y} = 0$. $S = I_N - X(X^T X)^{-1} X^T $ is a candidate for A, as $SX = 0$. Furthermore, it can be shown $A A^T = S$ and $A^T A = I_N$AT A = IN.

The error contrast vector 
\begin{equation}\label{eq1} 
         w = A^T y = A^T(X \beta + \epsilon)= A^T \epsilon \sim \mathcal{N}(0,A^T H A)
\end{equation}
 is free of $\beta$. [Patterson and Thompson, 1971] has proven that in the absence of information on $\beta$, no information about $\theta$ is lost when inference is based on $\mathcal{w}$ rather than on y.We can now directly estimate $\theta$ by maximizing a “restricted” log-likelihood function $\mathcal{L_w}(\theta| A^T y)$. This bypasses estimating $\beta$ first and can Therefore, produce unbiased estimates for $\theta$.e obtain exactly the same result as Equation.
\end{document}

