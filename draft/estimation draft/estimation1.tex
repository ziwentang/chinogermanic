\documentclass[12pt]{article}
\usepackage[utf8]{inputenc}
\usepackage{amsmath}
\usepackage{geometry}
\geometry{left=2cm,right=2cm,top=2cm,bottom=2cm}

\title{estimation part}
\author{Shanshan Song}
\date{16 December 2018}

\begin{document}

\maketitle

\section{Estimation analysis in simple linear model}
\subsection{Maximum likelihood estimation in linear model}
\subsubsection{ML estimator in linear model}
     $y = X\beta + \epsilon$
     \newcommand\iid{i.i.d.}
     \newcommand\pN{\mathcal{N}}
     \\$\epsilon \sim \pN(0,\sigma^2I_N)$
     \\Recall that $\hat{\beta} = ({X^TX})^{-1}X^Ty 
     \\{\hat{\sigma}}^2 = \frac1N(y-X\hat{\beta})^T(y-X\hat{\beta})$
    
\subsubsection{Estimation bias in variance component}
    $ E[{\hat{\sigma}}^ 2] = \frac{N-K}{N}\sigma^2<\sigma^2$
    \\The mathematical inference can be found in appendix. As we can see, estimator ${\hat{\sigma}}^ 2$ is biased downwards as compared with real value $\sigma^2$. This bias is especially severe when we have many regressors(a large K). The reason of bias is that we neglect the loss of \textbf{degree of freedom}(DoF) for estimating $\beta$.
\subsection{How to resolve bias?}
\subsubsection{Closed-form solutions existing}
    In simple problems where solutions to variance components are closed-form (like linear regression above), we can remove the bias post hoc by multiplying a correction factor. In example above, we need to correct this bias by simply multiplying a factor of $N/N-K$. Hence, the corrected, unbiased estimator becomes
    \begin{equation}
    \begin{aligned}
    {\hat{\sigma}}^2_{unbiased} &= \frac1{N-K}(y-X\hat{\beta})^T(y-X\hat{\beta})\\
    &=\frac1{N-K}(y-X({X^TX})^{-1}X^Ty)^T(y-X({X^TX})^{-1}X^Ty)
    \end{aligned}
    \end{equation}
\subsubsection{Closed-form solutions not existing}
    For complex problems where closed-form solutions do not exist, we
need to resort to a more general method to obtain a bias-free estimation for variance components. Generally, estimation bias in variance components originates from the DoF loss in estimating mean components. If we estimated variance components with true mean component values, the estimation would be unbiased. \textbf{Restricted Maximum Likelihood} is one such method.
\end{document}
