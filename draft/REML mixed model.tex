\documentclass[a4paper,11pt]{article} 
% Load some standard packages
\usepackage{amsmath,parskip,fullpage,natbib,bm} 

% Load tikz for the tree diagram
\usepackage{tikz}
\usetikzlibrary{trees,shapes}

% Define my own commands
\newcommand{\info}{{\cal I}}
\newcommand{\E}{\text{E}}

% Bibliography style
\bibliographystyle{agsm}

\begin{document}  
\title{Mixed effects}
\author{Lucas}
\maketitle

\section{Estimation by Restricted Maximum Likelihood:}

In this part we want to estimate values for the fixed effects coefficient $\beta$ and use the data to model $\sigma^2$ and the matrix $D$. Because the solutions for $\theta$ are not closed form, $D$ and $\sigma^2$ will not be either. As explained above, ML estimates are biased in this case, even after a post hoc correction. Thus we have to apply the restricted maximum likelihood method, instead of adjusted ML values. To do this, we rewrite \label{basic} in matrix notation:
\begin{equation}
\bm{y}=\bm{X}\beta + \bm{Zb}+\epsilon	
\end{equation}

Denote $\bm{y}$=$\left[\begin{array}{c}
y_1 \\
y_2 \\
$\vdots$ \\
y_M \end{array} \right]$,      $\bm{X}$ = $ \left[ \begin{array}{c}
X_1 \\
X_2 \\
$\vdots$ \\
X_M \end{array} \right]$,      $\bm{Z}$ = $ \left[ \begin{array}{cccc}
Z_1 & 0 & \cdots & 0 \\
0 & Z_2 & \cdots& 0 \\
\vdots & \vdots & \ddots& \vdots\\
0 & 0& \cdots & Z_M \end{array} \right]$,	$\bm{b} $= $ \left[ \begin{array}{c}
b_1 \\
b_2 \\
$\vdots$ \\
b_M \end{array} \right]$,      $\epsilon$ = $ \left[ \begin{array}{c}
\epsilon_1 \\
\epsilon_2 \\
$\vdots$ \\
\epsilon_M \end{array} \right]$. 

Thus, with $\bm{H}(\theta$) being a diagonal matrix we get \begin{equation}
	\bm{y}\sim\mathcal{N}\biggl( \bm{X}\beta,\bm{H}(\theta)= \left[ \begin{array}{cccc}
	H_1(\theta) & 0 & \cdots & 0 \\
	0 & H_2(\theta)& \cdots& 0 \\
	\vdots & \vdots & \ddots& \vdots\\
	0 & 0& \cdots & H_M(\theta) \end{array} \right] \biggr),	
\end{equation}

Using this and dropping constant terms we get a new version of Equation XY:

\begin{equation}\label{eq:2} 
\begin{split}
{\mathcal{L}}_w (\theta| A^T y) &= -\frac{1}{2}\text{log det}H-\frac{1}{2}\text{log det} X^TH^{-1}X-\frac{1}{2}(y-X\widehat{\beta})^TH^{-1}(y-X\widehat{\beta}) \\
&= -\frac{1}{2}\text{log det}\bm{H}-\frac{1}{2}\text{log det}\bm{X}^T\bm{H}^{-1}\bm{X}-\frac{1}{2}(\bm{y}-\bm{X}\widehat{\beta})^TH^{-1}(\bm{y}-\bm{X}\widehat{\beta}) \\
&= -\frac{1}{2}\text{log}\prod_{i=1}^{M} \text{det}H_i-\frac{1}{2}\text{log}\prod_{i=1}^{M} \text{det} X_i^TH_i^{-1}X_i-\frac{1}{2}\sum_{i=1}^{M}(y_i-X_i\widehat{\beta})^TH_i^{-1}(y_i-X_i\widehat{\beta})\\
&= -\frac{1}{2}\sum_{i=1}^{M} \text{log det}H_i-\frac{1}{2}\sum_{i=1}^{M} \text{log det} X_i^TH_i^{-1}X_i-\frac{1}{2}\sum_{i=1}^{M}(y_i-X_i\widehat{\beta})^TH_i^{-1}(y_i-X_i\widehat{\beta})
\end{split}
\end{equation}
This equation can be used to get estimates of $\theta$ by maximizing it w.r.t. $\beta$, $\sigma$ and D.\footnote{for computational Details check appendix}
As shown previously??! 
\begin{equation}
\begin{split}
\widehat{\beta}&=(\bm{X}^T\bm{H}^{-1}\bm{X}^T\bm{H}^{-1}\bm{y})\\
&=\bigl(\sum_{i=1}^{M}X_i^TH_i^{-1}X_i\bigr)^{-1}\sum_{i=1}^{M}X_i^TH_i^{-1}y_i.
\end{split}
\end{equation}
\end{document}

