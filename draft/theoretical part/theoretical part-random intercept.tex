\documentclass[12pt]{article}
\usepackage[utf8]{inputenc}
\usepackage{appendix}
\usepackage{graphicx}
\usepackage{amsmath}
\usepackage{geometry}
\usepackage{setspace}
\usepackage{float}
\usepackage{subfigure}
\usepackage[]{caption2}
\renewcommand{\baselinestretch}{1.5}
\newcommand\iid{i.i.d.}
\newcommand\pN{\mathcal{N}}
\geometry{left=2cm,right=2cm,top=2cm,bottom=2cm}

\title{}
\author{Ziwen Tang}
\date{17 February 2019}

\begin{document}

\maketitle
\section{Random intercept model}

\subsection{Why would random intercept model be considered?}
In the clustered data observations within the same clusters share the common correlated feature which is not considered in classical linear model.
Because of independent and homoscedastic residuals, the classical linear model cannot handle intracorrelated responses $y_{ij}$ and $y_{il}$ on the same cluster i appropriatelly.In a word, there exists cluster-specific heterogeneity shared among the measurements in the same cluster.How to use a more precise model to take such heterogeneity explicitly into account? 

First let us have a look at the simplest type of cluster-specific heterogeneity.
\begin{figure}[ht]
\centering
\includegraphics[scale=0.6]{figure.png}
\caption{Illustration of random intercept models: both panels show separately estimated regression lines for each cluster. In the left panel, there is no cluster-specific random intercept, while in
the right panel a random intercept is present. The dashed line corresponds to the population model}
\label{fig:label}
\end{figure}

The plots describe the model $y_{ij} = 1 + x_{ij} +\epsilon_{ij}$, , j = 1,…,20, for subject j = 1,…,20 in cluster i=1,…,10, errors $\epsilon_{ij} \stackrel{\text{i.i.d}}{\sim} N(0,0.1^2)$. The dashed line shows the population model. The other lines are drawn from the estimated cluster-specific intercept regression models sharing a common slope across clusters.

\subsection{Random Intercept Model Construction}
The estimated cluster-specific regression lines still show a common slope across clusters, but the intercept appears to be different from cluster to cluster.

To take correlations between observations of the same cluster 
into account (at least to some extent), we introduce this type of cluster-specific parameters $\gamma_{0i}$ and obtain equation
\begin{equation}\label{4}
Y_{ij} = \beta_0 + \beta_1x_{ij}+\gamma_{0i}+\epsilon_{ij}
\end{equation}


$\beta_0$ is the fixed population intercept.

$\beta_1$ is the fixed population slope.

$\gamma_{0i}$ is the random intercept,also the cluster-specific deviation from the fixed population intercept $\beta_0$, assuming clusters are randomly
sampled from a larger population and  $\gamma_{0i} \stackrel{\text{i.i.d}}{\sim}  N(0,\tau_0 ^2)$

$\epsilon_{ij}\stackrel{\text{i.i.d}}{\sim} N(0,\sigma^2)$ 

$\beta_0 + \gamma_{0i}$ is the cluster i's mixed intercept to correct for the simpest unobserved heterogeneity induced by omitted covariates., which is randomed with $N (\beta_0,\tau_0 ^2)$.

Assuming that $\epsilon_{ij}$ and $\gamma_{0i}$ is mutually independent, the random intercept model takes $\epsilon_{ij}$ and $\gamma_{0i}$ as  two error terms, where $\gamma_{0i}$ is a within-cluster error and $\epsilon_{ij}$ is a all-obversation error.

Formally, the introduction of random intercept in linear mixed effect models (LMMs) helps to distinguish the conditional (specific cluster-average) mean $E(y_{ij}| \gamma_{0i}) = \beta_0 + \beta_1 x_{ij} + \gamma_{0i}$  and marginal (population-average) mean $ E(y_{ij})= \beta_0 + \beta_1 x_{ij} $ as well as the conditional covariance $Cov(y_{ij}| \gamma_{0i})= \sigma^2 $and marginal covariance $ Cov(y_{ij})= \sigma^2 + \tau_0^2$
\end{document}
