\documentclass[12pt]{article}
\usepackage[utf8]{inputenc}
\usepackage{appendix}
\usepackage{graphicx}
\usepackage{amsmath}
\usepackage{geometry}
\usepackage{setspace}
\usepackage{float}
\usepackage{subfigure}
\usepackage[]{caption2}
\renewcommand{\baselinestretch}{1.5}
\newcommand\iid{i.i.d.}
\newcommand\pN{\mathcal{N}}
\geometry{left=2cm,right=2cm,top=2cm,bottom=2cm}

\title{}
\author{Shanshan Song}
\date{10 February 2019}

\begin{document}

\maketitle
\section{Random Slope model}
\subsection{Why would random slope model be considered?}
From the random intercept model discussed above, we know that it is based on the assumption that the relationship between the covariate and response is the same in every group. However, sometimes the effect of the covariate may differ from group to group and this may be of interest. One case is that the intercept for the regression lines looks identical(or at least close) but the slopes are obviously different, as Fig 1 (i) in Appendix shows. A more typical case is that all regression lines show both cluster-specific intercepts and slopes, as Fig 1 (ii) shows.(\cite{Fahrmeir})
\subsection{Random Slope Model Construction}
\begin{equation}
    y_{ij} = \beta_0 + \beta_1x_{ij} + b_{0i} + b_{1i}x_{ij} + \epsilon_{ij} 
\end{equation}
where:
\\ $\beta_1$ is the "fixed" population slope of the effect of $x$.
\\ $b_{1i}$ is the individual- or cluster-specific variability from the slope.
\\ $\beta_1x_{ij}$ is the population effect of $x$.
\\ $\beta_1x_{ij} + b_{1i}x_{ij}$ is the cluster-specific effect of $x$.

\begin{flalign}
&\boldsymbol{b}_i = \left(
\begin{array}{c}
     b_{0i}\\
     b_{1i}
\end{array}
\right)
\overset{i.i.d.}{\sim} \pN \left( \left(
\begin{array}{c}
     0\\
     0
\end{array}
\right),  \left(
  \begin{array}{cc}
          \tau_0^2 & \tau_{01}\\
          \tau_{10} & \tau_1^2\\
  \end{array}
  \right)
  \right) 
 , \quad \left(
  \begin{array}{cc}
          \tau_0^2 & \tau_{01}\\
          \tau_{10} & \tau_1^2\\
  \end{array}
  \right) = \boldsymbol{D} \end{flalign}
We can see that cluster-specific parameter vector $\boldsymbol{b}_i$ has a bivariate normal random effects distribution with mean vector \boldsymbol{0} and symmetric covariance matrix \boldsymbol{D}. The variances of $b_{0i}$ and $b_{1i}$ are $\tau_0^2$ and $\tau_1^2$ respectively, which determine the variability of cluster-specific intercepts and slopes.
\section{Correlation structure on observations}
We firstly get the marginal heteroscedastic variance of $y_{ij}$
\begin{equation}
    Var(y_{ij}) = \tau_0^2 + 2\tau_{01}x_{ij} + \tau_1^2{x_{ij}}^2 + \sigma^2
\end{equation}
See a derivation in Appendix. Covariate $x_{ij}$ can affect the variance of $y_{ij}$. The covariance between $y_{ij}$ and $y_{ir}$ is shown to be(\cite{Fahrmeir})
\begin{equation}
    Cov(y_{ij},y_{ir}) = \tau_0^2 + \tau_{01}x_{ij} + \tau_{01}x_{ir} + \tau_1^2x_{ij}x_{ir}
\end{equation}
In a further step we get an intraclass correlation coefficient
\begin{equation}
    Corr(y_{ij},y_{ir}) = \frac{Cov(y_{ij},y_{ir})}{Var(y_{ij})^{1/2}Var(y_{ir})^{1/2}}
\end{equation}
This coefficient has a rather complex form depending on observed $x$ values and we cannot interpret easily.
\section{Appendix}
\subsection{}
\renewcommand{\figurename}{Fig.} 
\renewcommand{\captionlabeldelim}{.~} 
\renewcommand{\thesubfigure}{(\roman{subfigure})}
\makeatletter \renewcommand{\@thesubfigure}{\thesubfigure \space}
\renewcommand{\p@subfigure}{} \makeatother
\begin{figure}[H]
\centering  
\subfigure[]{
\label{Fig.sub.1}
\includegraphics[width=0.45\textwidth]{1.png}}
\subfigure[]{
\label{Fig.sub.2}
\includegraphics[width=0.45\textwidth]{2.png}}
\caption{Random Slope Model}
\label{Fig.main}
\end{figure}
The Fig 1 (i) shows that it´s a model where most random intercept parameters are zero or close to zero and random slope parameters clearly different from zero. There is nearly no variability among cluster-specific intercepts, but there is distinct variability among cluster-specific slopes. (ii) shows that it´s a model where both random intercept parameters and random slope parameters clearly different from zero. There exists variability among both cluster-specific intercepts and slopes.
\subsection{}
Rewrite the model in matrix notation as 
\begin{equation}
    \boldsymbol{y}_i = \boldsymbol{X}_i\boldsymbol{\beta} + {\boldsymbol{\epsilon}_i}^*
\end{equation}
with errors ${\boldsymbol{\epsilon}_i}^*$ = $\boldsymbol{Z}_i\boldsymbol{b}_i$ + $\boldsymbol{\epsilon}_i$, results in a linear Gaussian regression model with correlated errors
\\ ${\boldsymbol{\epsilon}_i}^*$ \sim \pN(\boldsymbol{0}, $\boldsymbol{H}_i$), $\boldsymbol{H}_i$ = $Cov(\boldsymbol{\epsilon}_i)$ + $Cov(\boldsymbol{Z}_i\boldsymbol{b}_i)$ = $\sigma^2\boldsymbol{I}$ + $\boldsymbol{Z}_i\boldsymbol{D}{\boldsymbol{Z}_i}^T$.
\\The first equality for $\boldsymbol{H}_i$ holds because $\boldsymbol{\epsilon}_i$ and $\boldsymbol{b}_i$ are assumed to be independent and $Cov(\boldsymbol{Z}_i\boldsymbol{b}_i)$ = $\boldsymbol{Z}_i\boldsymbol{D}{\boldsymbol{Z}_i}^T$ follows from $Cov(\boldsymbol{AX} + \boldsymbol{b}) = \boldsymbol{A}Cov(\boldsymbol{X}){\boldsymbol{A}}^T$ given \boldsymbol{A}, \boldsymbol{b} matrices or vectors of appropriate order and \boldsymbol{X} a random vector. Thus the corresponding marginal Gaussian model for $\boldsymbol{y}_i$ is given by 
$\boldsymbol{y}_i \sim \pN(\boldsymbol{X}_i\boldsymbol{\beta}, \sigma^2\boldsymbol{I}+\boldsymbol{Z}_i\boldsymbol{D}{\boldsymbol{Z}_i}^T)$, it's not difficult to imply that $Var(y_{ij}) = \tau_0^2 + 2\tau_{01}x_{ij} + \tau_1^2{x_{ij}}^2 + \sigma^2$.(\cite{Fahrmeir})



      
 
\end{document}
